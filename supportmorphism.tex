\documentclass[letterpaper,10pt]{article}

\setlength{\evensidemargin}{-0.0in}
\setlength{\oddsidemargin}{-0.0in}
\setlength{\textwidth}{6.4in}

\usepackage[utf8]{inputenc}
\usepackage{amsmath}
\usepackage{amssymb}
\usepackage{enumerate}
\usepackage{mathpazo}
\usepackage{relsize}
\usepackage{mathtools}
\usepackage{mathrsfs}
\usepackage{verbatim}

\usepackage{hyperref}
\hypersetup{
    colorlinks=true,
    linkcolor=blue,
    filecolor=magenta,      
    urlcolor=blue,
}

\usepackage{tikz}
\usetikzlibrary{matrix,arrows}

\usepackage{fancyhdr}
\pagestyle{fancy}
\lhead{Support morphism}
\rhead{Tuomas Tajakka}
%\cfoot{}
\rfoot{}
\renewcommand{\headrulewidth}{0.4pt}

\usepackage{amsthm}

\newtheorem{thm}{Theorem}[section]
\newtheorem{lem}[thm]{Lemma}
\newtheorem{prop}[thm]{Proposition}
\newtheorem{cor}[thm]{Corollary}
\newtheorem{expl}[thm]{Example}
\newtheorem{defn}[thm]{Definition}

\theoremstyle{remark}
\newtheorem{rmk}[thm]{Remark}
\newtheorem{exer}[thm]{Exercise}

\title{Support morphism}
\author{Tuomas Tajakka}
\date{}

\usepackage{commands}


\begin{document}

\maketitle

\section{Stuff}
Let $X$ be a smooth, projective, irreducible surface, and let $H \subs X$ be a very ample divisor. Let $v \in K(X)$ be a class such that $\ch_0(v) = 0$ and $\ch_1(v) \cdot H > 0$. Let $\si \in \Stab(X)$ lie in the Simpson chamber for $v$. According to Wanmin, there is a line bundle $\sS$ on $\sM^\si(v)$ given by 
\[ \sS = \la_\sE(\Oh_x) = \det(R p_*(\sE \otimes^\LL q^* \Oh_x)) = \det(\sE|^\LL_{\sM^\si(v) \times \{x\}}) \]
that should play a role analogous to the line bundle $\sL_1$ on the Gieseker moduli space, and give a "support morphism" from $\sM^\si(v)$ to somewhere. 

It is feasible that the line bundle $\sS$ is semiample on $\sM^\si(v)$. Using determinantal line bundle techniques, we need to check that the complex $\sE|^\LL_{\sM^\si(v) \times \{x\}}$ has cohomology sheaves in degrees $-1$ and $0$, and that for any closed point $t \in \sM^\si(v)$ we can choose $x \in X$ such that (i) the class $[\Oh_x] \in K(X)$ remains unchanged, and (ii) the cohomology of $\sE|^\LL_{\sM^\si(v) \times \{x\}}$ vanishes at $t$.

We can show that $\sE|^\LL_{\sM^\si(v) \times \{x\}}$ is indeed concentrated in degrees $-1$ and $0$. By Cohomology and Base Change, it is enough to show that $H^i(X, \sE_t \otimes^\LL \Oh_x) = 0$ for any $t 
\in \sM^\si(v)$ and $x \in X$ whenever $i \neq -1, 0$. Now $\sE_t$ is pure of dimension 1. If $x \notin \Supp(\sE_t)$, then $\sE_t \otimes^\LL \Oh_x = 0$ and the claim is clear. If $x \in \Supp(\sE_t)$, then $\sE_t$ has depth 1 as a module over the regular local ring $\Oh_{X,x}$, hence homological dimension 1 over $\Oh_{X,x}$ by the Auslander-Buchsbaum formula \cite[equation (1.1)]{HL}. Thus, $\sE_t$ can be resolved by a 2-term complex of free modules over $\Oh_{X,x}$, and so $\sE_t \otimes^\LL \Oh_x$ is supported in degrees $-1$ and $0$.

%To achieve the second step, we note that $H^0(X, \sE_t \otimes^\LL \Oh_x) = 0$ if and only if $x \notin \Supp(\sE_t)$. Thus, we must need to choose $x \in X$ so that for any $t \in \sM^\si(v)$, the class $[\Oh_x] \in X$ contains $\Oh_y$ for some $y \notin \Supp(\sE_t)$.

%We observe that if $x, y \in X$ are contained in a smooth rational curve $C \subs X$, then $[\Oh_x] = [\Oh_y]$ in $K(X)$. This holds because $[\Oh_x] = [\Oh_y]$ in $K(C)$ and the pushforward $i_*: \Coh(C) \to \Coh(X)$ is exact. Thus, it would be enough to show that $X$ contains a smooth rational curve $C$ of arbitrarily high degree. If this is the case, then we can choose $C$ so that it cannot be contained in $\Supp(\sE_t)$ for any $t \in \sM^\si(v)$, because the degree of $\Supp(\sE_t)$ is bounded.





\bibliographystyle{siam}
\addcontentsline{toc}{section}{References}
\bibliography{bibliography}

\end{document}
